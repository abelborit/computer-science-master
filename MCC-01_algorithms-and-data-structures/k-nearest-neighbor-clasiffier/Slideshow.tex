\documentclass [xcolor=svgnames, t] {beamer} 
\usepackage[utf8]{inputenc}
\usepackage{booktabs, comment} 
\usepackage[absolute, overlay]{textpos} 
\useoutertheme{infolines} 
\setbeamercolor{title in head/foot}{bg=internationalorange}
\setbeamercolor{author in head/foot}{bg=dodgerblue}
\usepackage{csquotes}
\usepackage[style=verbose-ibid,backend=bibtex]{biblatex}
\bibliography{bibfile}
\usepackage{amsmath}
\usepackage[makeroom]{cancel}
\usepackage{textpos}
\usepackage{tikz}
\usetheme{Madrid}
\definecolor{myuniversity}{RGB}{0, 60, 113}
\definecolor{internationalorange}{RGB}{231, 93,  42}
 	\definecolor{dodgerblue}{RGB}{0, 119,202}
\usecolortheme[named=myuniversity]{structure}
\usepackage{tikz}

\institute[UNSA]{\large \textbf{Universidad Nacional de San Agustín }\\ Maestría en Ciencias de la Computación \\ Algoritmos y Estructuras de Datos}

\title[Trabajo de Investigación]{Análisis de viabilidad crediticia mediante KD-Tree y KNN} 
\subtitle{Un enfoque de Aprendizaje Automático para evaluar el riesgo de pago de nuevos solicitantes}

\author[Abel E. Borit, Luis A. Borit, Betzy J. Yarín]{Abel E. Borit Guitton, Luis A. Borit Guitton, Betzy J. Yarín Ramírez} 

\date{Agosto, 2023}


\addtobeamertemplate{navigation symbols}{}{%
    \usebeamerfont{footline}%
    \usebeamercolor[fg]{footline}%
    \hspace{1em}%
    \insertframenumber/\inserttotalframenumber
}

\begin{document}
\begin{frame}
 \titlepage   
\end{frame}

%%%%%%%%%%%%%%%%%%%%%%%%%%%%
\logo{\includegraphics[scale=0.08]{Img/Escudo_UNSA.png}~%
}
%%%%%%%%%%%%%%%%%%%%%%%%%%

\begin{frame}{Índice}
\vspace{1cm}
\begin{center}
   \begin{itemize}
     \item Introducción
     \item Problema
     \item Objetivos
     \item Implementación del KD-Tree y KNN
     \item Métricas
     \item Conclusiones
     \item Referencias
 \end{itemize} 
\end{center} 
\end{frame}

\begin{frame}{Introducción}
    La relevancia de KNN y KD-Tree en diversos contextos:
    \begin{center}    
    \begin{columns}[onlytextwidth,T]
    \column{0.5\textwidth}
    \begin{itemize}
    \item Identificación de patrones:
    \begin{itemize}
        \item Detectan similitudes y estructuras en datos complejos.
    \end{itemize}
    \item Eficiencia en búsquedas:
    \begin{itemize}
        \item Optimizan la búsqueda de vecinos cercanos en conjuntos de datos extensos.
    \end{itemize}
    \item Análisis exploratorio:
    \begin{itemize}
        \item Facilitan la exploración y comprensión de conjuntos de datos multidimensionales.
    \end{itemize}
\end{itemize}

\column{0.5\textwidth}
\begin{figure}
    \centering
    \includegraphics[width=0.7\textwidth]{Img/KDtree_KNN.png}
    \caption{1. Kd-Tree y KNN}
  %  \label{fig:my_label}
\end{figure}
\end{columns}
\end{center} 
\end{frame}

\begin{frame}{Problema}
\begin{center}
%\vspace{1 cm}
\textbf{Clasificación de solicitantes de crédito utilizando KD-Tree y K-Nearest Neighbors (KNN)}\\
\vspace{0.8cm}
\begin{itemize}
\item  La implementación utilizará los datos de edad y monto de crédito de clientes que pagaron y no pagaron préstamos en el pasado. Mediante el algoritmo K-Nearest Neighbors (KNN) y una estructura KD-Tree, clasificará nuevos solicitantes en estas categorías. Se requiere tomar decisiones informadas sobre aprobaciones de crédito basadas en similitudes con el historial de pagos.
\end{itemize}
\end{center}
\end{frame}

\begin{frame}{Objetivos}
 \vspace{1cm}   
\begin{itemize}
    \item Implementar KNN y KD-Tree: Optimizar la clasificación de nuevos solicitantes de crédito basados en características de edad y monto de crédito.
    \item Evaluar la viabilidad crediticia basada en similitudes con historiales de pagos anteriores.
    \item Mostrar los resultados de la clasificación en un gráfico que destaque la posición de los solicitantes en función de sus características de edad y monto de crédito.
\end{itemize}
\end{frame}
\begin{frame}
{KD-Tree y KNN\autocite{GeorgeGaryStanley2009Algorithms}}
\begin{figure}
    \centering
    \includegraphics[width=0.85\textwidth]{Img/ResumenKNN_KDTree.png}
    %\caption{}
  %  \label{fig:my_label}
\end{figure}
\end{frame}

\begin{frame}{Implementación KD-Tree\autocite{Bishop2006PRML}}
\begin{figure}
    \centering
    \includegraphics[width=0.9\textwidth]{Img/AlgoritmoKD-Tree.png}
    \caption{2. Pseudocódigo KD-Tree}
  %  \label{fig:my_label}
\end{figure}  
\end{frame}

\begin{frame}{Implementación KNN\autocite{Alpaydin2014IntroML}}
\begin{figure}
    \centering
    \includegraphics[width=0.75\textwidth]{Img/AlgoritmoKNN.png}
    \caption{3. Pseudocódigo KNN}
  %  \label{fig:my_label}
\end{figure} 
\end{frame}


\begin{frame}{Métricas}
    \begin{center}    
    \begin{columns}[onlytextwidth,T]
    \column{0.5\textwidth}
    \begin{itemize}
        \item Cada una se calcula utilizando diferentes fórmulas. F1-Score depende de las primeras métricas y busca encontrar un equilibrio entre ellas. 
       \item El resultado de las métricas depende de la evaluación del modelo y de los datos utilizados para el entrenamiento y prueba. 
       \item El tener valores iguales en las métricas es un indicio que el modelo está obteniedo resultados consistentes, también se le atribuye a la sencillez del modelo.
    \end{itemize} 

\column{0.5\textwidth}
    \begin{figure}
        \centering
        \includegraphics[width=0.9\textwidth]{Img/MetricasEvaluacion.png}
        \caption{4.Desempeño del Modelo}
      %  \label{fig:my_label}
    \end{figure}
\end{columns}
\end{center} 
\end{frame}


\begin{frame}{Conclusiones}

\begin{itemize}
    \item La implementación de KNN y KD-Tree para evaluar el riesgo crediticio demuestra ser eficaz para clasificar nuevos solicitantes en base a características como edad y monto de crédito.
    \item La estructura KD-Tree mejora la eficiencia en la búsqueda de vecinos cercanos, acelerando el proceso de clasificación.
    \item Los resultados muestran cómo los solicitantes son clasificados en función de la similitud con clientes anteriores, destacando el impacto de los vecinos cercanos en la decisión.
     \item El valor de k (número de vecinos más cercanos) se establece en 3 en la implementación. Elegir un valor óptimo para k es una decisión importante y puede afectar la precisión del modelo.
\end{itemize}    
\end{frame}
\begin{frame}{Referencias}
   \printbibliography
\end{frame}

\end{document}